\documentclass[12pt]{exam}
\usepackage[utf8]{inputenc}

\usepackage{graphicx}
\usepackage{makecell}
\usepackage{minibox}
\usepackage{multirow}
\usepackage{lastpage}
\usepackage{amsmath}
\usepackage[margin=0.5in]{geometry}
\usepackage{amssymb}
\usepackage{xcolor}
\usepackage{datetime}
\usepackage{amsmath}
\usepackage{mathtools}
\usepackage{tikz}
\usetikzlibrary{positioning}


\pagestyle{headandfoot}

\newdateformat{monthyeardate}{%
  \monthname[\THEMONTH], \THEYEAR}
%\newdateformat{\monthyeardate}{\monthname[\THEMONTH], \THEYEAR}




\newcommand{\paper}[5]{
    \setcounter{page}{1}
    
    \begin{minipage}{\textwidth}
    \Large{\textbf{Tougaloo College}}\\
    %\vspace{0.1in}\\  
   % Bachelor of the Science of Engineering \\
    %Academic Year #4\\
    %\vspace{0.1in}\\
    \textbf{#2}\\
    \textbf{#5 - #3}\\
    \end{minipage}
    \hfill
   % \begin{minipage}{3in}
      %  \centering
      %  \includegraphics[scale=0.23]{logo.jpg}
    %\end{minipage}\\
    %\hfill
   % \vspace{0.1in}\\
        %\begin{minipage}{6in}
         %   \textbf{\hspace{0.25in}Due Date : #1}
        %\end{minipage}
    
    \vspace{0.1in}
    \rule[1ex]{\textwidth}{2pt}
    }



\extrafootheight{.5in}
\cfoot{Page ~\thepage ~of ~\numpages}


\printanswers


\marginpointname{ \points}
\pointsinrightmargin
\pointpoints{ Point}{ Points}
\setlength{\rightpointsmargin}{3cm}
\pointsdroppedatright
\addpoints
%\qformat{\textbf{Question \thequestion} \hfill}

%%%%%%%%%%%%%%%%%%%%%%%%%%%%%%%%%%%%%%%%%%%%%%%%%%%%%%%%
%%%%%%%%%%%%%%%%%%%%%%%%%%%%%%%%%%%%%%%%%%%%%%%%%%%%%%%%


\begin{document}
%\paper{duration}{Course}{Semester}{academic year}{Examination}
\paper{03//2025}{MAT414 - Modern Algebra}{Spring, 2025}{}{Howework 05}

\section*{Finite Groups; Subgroups - Exercises}
\begin{questions}



\question[20] (Problem 35) Let \(G\) be a group. Show that \(Z(G)=\cap_{a \in G} C(a)\).

\begin{solution}
    Show that \(Z(G) \subset \cap_{a\in G}C_a\) and \(\cap_{a\in G} C(a) \subset Z(G)\).
\end{solution}
\droptotalpoints

\question[20] (Problem 36) Let \(G\) be a group and let \(a \in G\). Prove that \(C(a) = C(a^{-1}).\)

\section*{Cyclic Groups}

\question[] List the elements of subgroups \(\langle 3 \rangle\) and \(\langle  7 \rangle\) in \(U(20)\).

\begin{solution}
    \[ \langle 3 \rangle = \langle 7 \rangle= \{1,3,9,7\}\]
\end{solution}

\question[] Find an example of a non-cyclic group, all of  whose proper subgroups are cyclic. 

\begin{solution}
    \(Q_8\)
\end{solution}

\question[] In \(Z_{24}\), find a generator for \(\langle 21 \rangle \cap \langle 10 \rangle\). Suppose that \(|a|=24\). Find a generator for \(\langle a^{21} \rangle \cap \langle a^{10} \rangle\). In general, what is a generator for the subgroup \(\langle a^{m}\rangle \cap \langle a^n \rangle\)

\begin{solution}
    \(\langle a^m\rangle \cap \langle a^n \rangle = \langle a^{\operatorname{lcm}(m,n)} \rangle\)
\end{solution}


\question[] Suppose that a cyclic group \(G\) has exactly three subgroups: \(G\) itself, \(\{e\}\), and a subgroup of order 7. What is \(|G|\)? 

\begin{solution}
    \(|G|=49\)
\end{solution}

\question[] Determine the subgroup lattice for \(Z_{12}\). Generalize to \(Z_{p^2 q}\), where \(p\) and \(q\) are distinct primes. 

\begin{solution}
    


\begin{tikzpicture}[every node/.style={circle, draw, minimum size=1.5em, inner sep=2pt},
    level distance=1.5cm,
    sibling distance=3cm]

% Nodes
\node (top) {\(\langle 1 \rangle = \mathbb{Z}_{p^2 q}\)}
child { node (p) {\(\langle p \rangle\)}
child { node (p2) {\(\langle p^2 \rangle\)} }
child { node (pq) {\(\langle pq \rangle\)} }
}
child { node (q) {\(\langle q \rangle\)}
%child [grow=down] { node [below=0.7cm of q] (pq2) {} }
};

% Bottom node
\node (zero) [below=3cm of top] {\(\langle 0 \rangle\)};

% Connections
\draw (p2) -- (zero);
\draw (pq) -- (zero);
\draw (q) -- (pq);

\end{tikzpicture}

\end{solution}


\end{questions}
\end{document}


