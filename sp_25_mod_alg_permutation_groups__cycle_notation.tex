\documentclass{beamer}

\usepackage{amsmath, amssymb}
\usepackage{tikz-cd}
\usepackage{xcolor}
\usepackage{graphicx}

\title{MAT414 - Modern Algebra - Permutation Groups}
\subtitle{Cycle Notation \cite{JAG2017}}
\author{\textbf{Miraj Samarakkody}}
\institute{Tougaloo College}
\date{Updated - \today}

\begin{document}

\begin{frame}
    \titlepage
\end{frame}




\begin{frame}
    \frametitle{Cycle Notation}

    Write the followings in the cyclic notations:
    \[
    \alpha = \begin{bmatrix}
        1 & 2 & 3& 4 & 5 & 6\\
        2& 1& 4 & 6 & 5 & 3
    \end{bmatrix}\quad \beta =  \begin{bmatrix}
        1 & 2 & 3& 4 & 5 & 6\\
        5& 3& 1 & 6 & 2 & 4
    \end{bmatrix}
    \] \pause
    Find \(\alpha \beta\). 

\end{frame}

\begin{frame}
    \frametitle{Properties of Permutations}

    \begin{block}{Theorem 5.1 - Products of Disjoint Cycles}
        Every permutation of a finite set can be written as a cycle or as a product of disjoint cycles. 
    \end{block}

\end{frame}

\begin{frame}
    \frametitle{Theorem 5.2}

    \begin{block}{Disjoint Cycles Commute}
        If the pair of cycles \(\alpha = (a_1,a_2, \dots, a_m)\) and \(\beta =(b_1, b_2, \dots, b_n)\) have no entries in common, then \(\alpha \beta = \beta \alpha\).
    \end{block}

\end{frame}

\begin{frame}
    \frametitle{Theorem 5.3}

    \begin{block}{Order of a Permutation}
        The order of a permutation of a finite set written in disjoint cycle form is the least common multiple of the lengths of the cycles. 
    \end{block}\pause

\vspace{0.2in}
    Find \begin{itemize}
        \item \(|(132)(45)|\)
        \item \(|(1432)(56)|\)
        \item \(|(123)(456)(78)|\)
        \item \(|(123)(145)|\)
    \end{itemize}

\end{frame}

\begin{frame}
    \frametitle{Example 5}

    Determine the orders of the elements of \(S_7\). 

\end{frame}

\begin{frame}
    \frametitle{Example 6}

    Determine the number of elements in \(S_7\) of order 12. 

\end{frame}

\begin{frame}
    \frametitle{Example 7}

    Determine the number of elements in \(S_7\) of order \(3\). 

\end{frame}

\begin{frame}
    \frametitle{Theorem 5.4}
    \begin{block}{Product of 2-Cycles}
          Every permutation of in \(S_n\), \(n>1\), is a product of \(2-\)cycles.   
    \end{block}

    

\end{frame}

\begin{frame}
    \frametitle{Example }

    \begin{align*}
        (1~2~3~4~5)=\\
        (1~6~3~2)(4~5~7)=
    \end{align*}

\end{frame}

\begin{frame}
    \frametitle{}
\begin{block}{Lemma}
    In \(S_n\), if \(\epsilon = \beta_1\beta_2 \beta_3 \dots \beta_r\), where the \(\beta_i\)'s are 2-cycles, then \(r\) is even. 
\end{block}
    

\end{frame}

\begin{frame}
    \frametitle{Theorem 5.5}

    \begin{block}{Always Even or Always Odd}
        If a permutation \(\alpha\) can be expressed as a product of an even (odd) number of 2-cycles, then every decomposition of \(\alpha\) into a product of \(2-\)cycles must have an an even (odd) number of \(2-\)cycles. \\
        \vspace{0.2in}
        In symbols, if \[\alpha = \beta_1 \beta_2 \dots \beta_r \text{ and } \alpha = \gamma_1 \gamma_2 \dots \gamma_s,\] where the \(\beta\)'s and \(\gamma\)'s are \(2-\)cycles, then \(r\) and \(s\) are both even or both odd.
    \end{block}

\end{frame}

\begin{frame}
    \frametitle{Even and Odd Permutations}
    \begin{block}{Definition}
        A permutation that can be expressed as a product of an even number of \(2-\)cycles is called an \textbf{even permutation}. A permutation that can be expressed as a product of an odd number of \(2-\)cycles is called an \textbf{odd permutation}.
    \end{block}
    

\end{frame}

\begin{frame}
    \frametitle{Even Permutations Form a Group} 
    \begin{block}{Theorem 5.6}
        The set of all even permutations of \(S_n\) is a subgroup of \(S_n\) and is denoted by \(A_n\).
    \end{block}
    

\end{frame}

\begin{frame}
    \frametitle{Alternating Group of Degree \(n\)}

    \begin{block}{Definition}
        The alternating group of degree \(n\), denoted \(A_n\), is the set of all even permutations of \(S_n\).
        
    \end{block}

\end{frame}

\begin{frame}
    \frametitle{Theorem}

    \begin{block}{Theorem 5.7}
        For \(n >1\), \(A_n\) has order \(n!/2\). 
    \end{block}

\end{frame}

\begin{frame}
    \frametitle{Example 10 - Rotations of a Tetrahedron}

    The 12 rotations of a regular tetrahedron can be conveniently described with the elements of \(A_4\). 
\includegraphics[scale=0.35]{Figures/fig_2.png}



\end{frame}

\begin{frame}
    \frametitle{Example 10 - Rotations of a Tetrahedron}

    The 12 rotations of a regular tetrahedron can be conveniently described with the elements of \(A_4\). 
\includegraphics[scale=0.5]{Figures/fig_3.png}



\end{frame}

\begin{frame}
    \frametitle{Applications}

    Many molecules with chemical formulas of the form \(AB_4\), such as methane \((CH_4)\) and carbon tetrachloride \((CCl_4)\), have \(A_4\) as thier symmetry group. 

    \includegraphics[scale=0.5]{Figures/fig_4.png}

\end{frame}







\begin{frame}
    \frametitle{References}
    \bibliographystyle{plain} % or another style like unsrt, alpha, etc.
    \bibliography{reference}  % omit the .bib extension
\end{frame}

\end{document}