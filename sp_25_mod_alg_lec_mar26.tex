\documentclass{beamer}

\usepackage{amsmath, amssymb}
\usepackage{tikz-cd}
\usepackage{xcolor}
\usepackage{graphicx}

\title{MAT414 - Modern Algebra}
\author{\textbf{Miraj Samarakkody}}
\institute{Tougaloo College}
\date{03/24/2025}

\begin{document}

\begin{frame}
    \titlepage
\end{frame}

\begin{frame}
    \frametitle{Cyclic Groups}

    \begin{block}{Theorem 4.2 - \(\langle a^k \rangle= \langle a^{\text{gcd(n,k)}}\rangle\) and \(|a^k|=n/ \text{gcd}(n,k)\)}
        Let \(a\) be an element of order \(n\) is a group and let \(k\) be a positive integer. Then \(<a^k>=<a^{\text{gcd}(n,k)}>\) and \(|a^k|=n/gcd(n,k)\)
    \end{block}\pause 
    \begin{block}{Proof steps:}
        We let \(d=\text{gcd}(n,k)\)
        \begin{itemize}
            \item In the first part, we have to prove \(<a^k> \subset <a^{\text{gcd}(n,k)}>\) and \(<a^k>\supset <a^{\text{gcd}(n,k)}>\)\pause
            \begin{itemize}
                \item Let \(d=nk\)\pause
                \item Write \(d=ns+kt\) for some integers \(s,t \)\pause
            \end{itemize} 
            \item Here we show that \(|a^d|\leq n/d\) and then \(|a^k| =n/d\). 
        \end{itemize}
        
    \end{block}
\end{frame}

    \begin{frame}
        \frametitle{Example 5}
    For \(|a|=30\), find \(< a^{26} >\) and \(|a|^{26}\).
        
    
    \end{frame}

    
    \begin{frame}
        \frametitle{Example 5}
    For \(|a|=30\), find \(< a^{17} >\) and \(|a|^{17}\).
        
    
    \end{frame}

    \begin{frame}
        \frametitle{Example 5}
    For \(|a|=30\), find \(< a^{18} >\) and \(|a|^{18}\).
        
    
    \end{frame}

    \begin{frame}
        \frametitle{Corollary 1}
        \begin{block}{Orders of Elements in Finite Cyclic Groups}
        In a finite cyclic group, the order of an element divides the order of the group. 
        \end{block}
    \end{frame}

    \begin{frame}{Corollary 2}
    \begin{block}{Criterian for \(\langle a^i\rangle = \langle a^j \rangle\) and \(|a^i|=|a^j|\)}   
    Let \(|a|=n\). Then \(\langle a^i \rangle = \langle a^j \rangle\) if and only if \(\text{gcd}(n,i) =\text{gcd}(n,j)\), and \(|a^i|=|a^j|\) if and only if \(\text{gcd}(n,i)=\text{gcd}(n,j)\).
    \end{block}
    \end{frame}

    \begin{frame}{Corollary 3}
    \begin{block}{Generators of Finite Cyclic Groups}
    Let \(|a|=n\). Then \(\langle a\rangle = \langle a^j\rangle\) if and only if \(\text{gcd}(n,j)=1\), and \(|a|=|\langle a^j\rangle|\) if and only if \(\text{gcd}(n,j)=1\).
    \end{block}
    \end{frame}

    \begin{frame}{Corollary 4}
    \begin{block}{Generators of \(\mathbb{Z}_n\)} 
    An integer \(k\) in \(\mathbb{Z}_n\) is a generator of \(\mathbb{Z}_n\) if and only if \(\text{gcd}(n,k)=1\).
    \end{block}
    \end{frame}

    \begin{frame}{Example}
    Find all generators of the cyclic group \(U(50)\).
    \end{frame}

    \begin{frame}{}
    \begin{center}
        \Huge{Fundamental Theorem of Cyclic Groups}
    \end{center}
    \end{frame}

    \begin{frame}{Theorem 4.3}
    \begin{block}{Fundamental Theorem of Cyclic Group}
        Every subgroup of a cyclic group is cyclic. Moreover, if \(|\langle a \rangle|=n\), then  the order of any subgroup of \(\langle a\rangle\) is a divisor of n; and, for each positive divisor \(k\) of \(n\), the group \(\langle a\rangle\) has exactly one subgroup of order \(k-\)namely, \(\langle a^{n/k}\rangle\). 
    \end{block}
    \end{frame}

    \begin{frame}{Example}
        Suppose \(G=\langle a\rangle\) and \(G\) has order 30. Find all the subgroups of \(G\).
    
    \end{frame}

    \begin{frame}
        \frametitle{Corollary}
        \begin{block}{Subgroups of \(\mathbb{Z}_n\)}
        For each positive divisor \(k\) of \(n\), the set \(\langle n/k\rangle\) is the unique subgroup of \(\mathbb{Z}_n\) of order \(k\); moreover, these are the only subgroups of \(\mathbb{Z}_n\).    
        \end{block}
    \end{frame}

    \begin{frame}
        \frametitle{Example 7}
    Write the list of subgroups of \(\mathbb{Z}_{30}\). 
    \end{frame}

\begin{frame}
    \frametitle{Example 8}
Find the generators of the subgroup of order 9 in \(\mathbb{Z}_{36}\). 
    

\end{frame}


\begin{frame}
    \frametitle{Euler Phi Function}
Let \(\phi(1)=1\), and for any integer \(n>1\), let \(\phi(n)\) denote the number of positive integers less than \(n\) and relatively prime to \(n\). 

\begin{block}{Example} Write each \(\phi(n)\) for \(n \in \{1,2, \dots, 12\}\)
    
\end{block}
    

\end{frame}




\end{document}